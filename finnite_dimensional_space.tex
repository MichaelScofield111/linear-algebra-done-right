\documentclass[a4paper,12pt]{article} 
\usepackage{amsmath}
\usepackage{ctex}
\usepackage{xcolor}


\begin{document}

\title {finite dimensional space}
% 生成标题
\maketitle

% 设置页码格式是罗马数字
\pagenumbering{roman}
% 生成目录
\tableofcontents
% 插入新页
\newpage
% 设置页码格式是阿拉伯数字
\pagenumbering{arabic}

\section{\textbf{span and linear independence}}
    \subsection{\textcolor{orange}{Notation} \textbf{list of vectors}}
    we will usually wirite lists of vectors without surrounding parentheses.

    \subsection{\textcolor{orange}{definition} \textbf{linear combination}}
    a \textbf{linear combination} of list $v_1,...,v_m$ of vectors in V is a vector of the form
    \begin{align*}
        a_1v_1 + ... + a_mv_m,
    \end{align*}
    where $a_1,...,a_m \in F$

    \subsection{\textcolor{orange}{definition} \textbf{span}}
    the set of all linear combination of a list of vectors $v_1,...,v_m$ in V is  called the span of 
    $v_1,...,v_m$, denoted span$(v_1,...,v_m)$. in other words,
    \begin{align*}
        span(v_1,...,v_m) = \{a_1v_1 + a_2v_2 + ... + a_mv_m : a_1,...,a_m \in F\}.
    \end{align*}
    the span of the empty list () is defined to be \{ 0\}

    \subsection{\textcolor{blue}{span is the smallest containing subspace}}
    the span of a list of vectors in V is the smallest subspace in of V containing all
    the vectors in the list.
    \subsection{definition \textbf{spans}}
    if span $(v_1,...,v_m)$ equals V, we say that $v_1,...,v_m$ spans V.
    \subsection{definition \textbf{polynoomial,} $\rho(F)$}
    \begin{itemize}
        \item A function $\rho$ : F -> F is called a polynoomial with coefficients in F if 
        there exists $a_0,...,a_m \in F$ sunch that
        \begin{align*}
            \rho(z) = a_o + a_1z + a_2z^2 + .... +a_mz^m
        \end{align*} 
        for all z $\in$ F
        \item $\rho$ is set of all polynoomial with coefficients $\in$ F
    \end{itemize}
    \subsection{\textbf{degree of a polynoomial,} deg(p)}
    \begin{itemize}
        \item A polynoomial $p \in \rho(F)$ is said to hve degree m if 
        there exists scalars $a_0,...,a_m \in F$ with $a_m \neq 0$ sunch that
        \begin{align*}
            \rho(z) = a_0 + a_1z +...+ a_mz^m
        \end{align*}
        for all $z \in F$ if p has degree m, we write deg(p) = m
        \item the polynoomial that is identically 0 is said to have degree $-\infty$
    \end{itemize}
    \subsection{\textcolor{blue}{Linear independence}}
    \begin{itemize}
        \item A list $v_z,...,v_m$ of vectors in V is called \textbf{linearly independent} if the 
        only choice of $a_1,...,a_m \in F$ that makes $a_1v_1 + ... + a_mv_m$ equal 0 is $a_1 = ... = a_m = 0$
        \item the empty list() is also declared to be linearly independent
    \end{itemize}
    \subsection{\textbf{linearly dependent}}
    \begin{itemize}
        \item A list of vectors in V is called linearly dependent if it is not linearly independent.
        \item  in other words, a list $v_1,...,v_m$ of vectors in V is linearly dependent if there exists
        $a_1,...,a_m \in F$, not all 0, sunch that $a_1v_1 + ... + a_mv_m = 0$ 
    \end{itemize}
    \subsection{\textcolor{blue}{length of linearly independent list $\le$ length of spanning list}}
    In a finite-dimensional vector space, the length of every linearly independent list of vectors is less than
    or equal to length of every spanning list of vectors.
    \newpage
    \title {Bases}
    \subsection{\textbf{basis}}
    a \textbf{basis} of V is list of vectors in V that is linearly independent and spans V.
    \subsection{criterion for basis}
    A list of $v_1,...,v_n$ of vectors in V is a basis of V if and only if every $v \in V$ can be written uniquely in the form
    \begin{align*}
        v = a_1v_1 + ... + a_nv_n
    \end{align*}
    where $a_1,...,a_n \in F$
    \subsection{spanning list contains a basis}
    Every spanning list in a vector space can be reduced to a basis of the vector space.
    \subsection{linearly independent list extends to a basis}
    Every linearly independent list of vectors in a finnite-dimensional vector space can be extended to a basis of vector space.
    \subsection{every subspace of V is part of a direct sum equal to V}
    suppose V is finite-dimensional and U is subspace of V. then there is a subspace W of V sunch that V = U $\oplus$ w
    \subsection{Basis length does not depend on basis}
    any two bases of a finite-dimensional vector space have the same length.
    \subsection{\textcolor{orange}{Definition} \textbf{dimension}, dim$V$}
    \begin{itemize}
        \item The dimension of a finite-dimensional vector space is the length of any basis of the vector space
        \item the dimension of V (if  V is finite-dimensional) is denoted by dim$V$
    \end{itemize}
    \subsection{\textcolor{blue}{Linear independent list of the right length is a basis}}
    Suppose V is finite-dimensional. then every linearly independent list of vectors in V with length dim$V$ is a basis of V.
    \subsection{\textcolor{blue}{Spanning list of the right length is a basis}}
    Suppose V is finnite-dimensional. then every spanning list of vectors in V with length dim$V$ is basis of V.
    \subsection{\textcolor{blue}{Demension of a sum}}
    if $U_1$ and $U_2$ are subspace of a finite-dimensional vector space, then
    \begin{align*}
        dim(U_1 + U_2) = dimU_1 + dimU_2 - dim(U_1 \cap U_2).
    \end{align*}
\end{document}
