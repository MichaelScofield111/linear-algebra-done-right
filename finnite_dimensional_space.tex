\documentclass[a4paper,12pt]{article} 
\usepackage{amsmath}
\usepackage{ctex}
\usepackage{xcolor}


\begin{document}

\title {finite dimensional space}
% 生成标题
\maketitle

% 设置页码格式是罗马数字
\pagenumbering{roman}
% 生成目录
\tableofcontents
% 插入新页
\newpage
% 设置页码格式是阿拉伯数字
\pagenumbering{arabic}

\section{\textbf{span and linear independence}}
    \subsection{\textcolor{orange}{Notation} \textbf{list of vectors}}
    we will usually wirite lists of vectors without surrounding parentheses.

    \subsection{\textcolor{orange}{definition} \textbf{linear combination}}
    a \textbf{linear combination} of list $v_1,...,v_m$ of vectors in V is a vector of the form
    \begin{align*}
        a_1v_1 + ... + a_mv_m,
    \end{align*}
    where $a_1,...,a_m \in F$

    \subsection{\textcolor{orange}{definition} \textbf{span}}
    the set of all linear combination of a list of vectors $v_1,...,v_m$ in V is  called the span of 
    $v_1,...,v_m$, denoted span$(v_1,...,v_m)$. in other words,
    \begin{align*}
        span(v_1,...,v_m) = \{a_1v_1 + a_2v_2 + ... + a_mv_m : a_1,...,a_m \in F\}.
    \end{align*}
    the span of the empty list () is defined to be \{ 0\}

    \subsection{\textcolor{blue}{span is the smallest containing subspace}}
    the span of a list of vectors in V is the smallest subspace in of V containing all
    the vectors in the list.
\end{document}
