\documentclass[a4paper,12pt]{article} 
\usepackage{amsmath}
\usepackage{ctex}
\usepackage{xcolor}


\begin{document}

\title {vector space}
% 生成标题
\maketitle

% 设置页码格式是罗马数字
\pagenumbering{roman}
\pagenumbering{roman}
% 生成目录
\tableofcontents
% 插入新页
\newpage
% 设置页码格式是阿拉伯数字
\pagenumbering{arabic}


\section{$R^n$ and $C^n$}
\subsection{definition complex number}
    \begin{itemize}
        \item[*] \textbf{\textit{A complex number}} is an ordered pair (a,b), where $a, b \in R $, but we will write this as $a +bi$
        \item[*] the set of all complex numbers is denoted by C:
        \begin{align*}
           \notag C = \{ a + bi: a, b \in R \} . 
        \end{align*}
        \item[*]  \textbf{\textit{addition and multiplication}} on C are defined by \\
        \vspace{-2em}
        \begin{align*}
            \notag(a + bi) + (c + d)i &= (a + c) + (b + d)i, \\
            \notag(a + bi) * (c + d)i &= (ac - bd) + (ab + bc)i, \\
        \notag here a,b,c,d \in R.
        \end{align*}
    \end{itemize}
\subsection{Properties of complex arithmetic}
\textbf{commutativity}
\begin{align*}
    \notag \alpha + \beta = \beta + \alpha  \text{ and }  \alpha\beta = \beta\alpha \text{ for all } \alpha,\beta \in R
\end{align*}
\textbf{associativity} 
\begin{align*}
    \notag (\alpha + \beta) + \lambda = \alpha + (\beta + \lambda) \text{ and }  (\alpha\beta)\lambda = \alpha(\beta\lambda)  \text{ for all } \alpha,\beta,\lambda \in R
\end{align*}
\textbf{indentities} 
\begin{align*}
    \lambda + 0 \text{ and } \lambda1 = \lambda \text{ for all } \lambda \in C;
\end{align*}
\textbf{indentities} 
\begin{align*}
    \text{for every } \alpha \in C, \text{ there exists a unique } \beta \in C \text{ such that } \alpha + \beta = 0;
\end{align*} 
\textbf{multiplicative inverse}
\begin{align*}
    \text{for every} \alpha \in C, \text{ with } \alpha \ne 0, \text{ there exists a unique } \beta \in C \text{ such that } \alpha\beta = 1; 
\end{align*} 

\subsection{Notation F}
Throughout this book,F stand for either R or C

\subsection{Example}
\begin{itemize}
    \item The set $R^3$, which you can think of as ordinary space, is the set of all ordered triples of real numbers:
    \begin{align*}
        R^3 = \{(x, y, z): x, y ,z \in R\}
    \end{align*}
\end{itemize}


\subsection{definition \textbf{list,length}}
Suppose n is a nonnegative interger. A \textbf{list} of \textbf{length} n is an ordered collection of n elements(witch might be numbers, other
lists, or more abstract entities) separated by commas and surrounded by parentheses. A list of length n looks like this:
\begin{align*}
    (x_1,.....,x_n)
\end{align*}
Two lists are equal if and only if they have the same length and the some elements in the same order.

\subsection{definition $F^n$}
$F^n$ is the set of all lists of length n of elements of F:
\begin{align*}
    F^n = \{(x_1,...,x_n):x_j \in R for j = 1,...n\}
\end{align*}
For $(x_1,...,x_n) \in F^n$ and $j \in \{1,...,n\}$, we say that $x_j$ is the $j^{th}$ coordinate of $(x_1,...,x_n)$ 
    \subsubsection{definition in $F^n$}
        \begin{align*}
            (x_1,...,x_n) + (y_1,...y_n) = (x_1 + y_1,...,x_n + y_n)
        \end{align*}
    \subsubsection{scalar multiplication in $F^n$}
    The product of a number $\lambda$ and  a vector in $F^n$ is computed by multiplying each coordinate of the vector by $\lambda$:
        \begin{align*}
            \lambda(x_1,...,x_n) = (\lambda x_1,...,\lambda x_n)
        \end{align*}
    here $\lambda \in F$ and $(x_1,...,x_n) \in F^n$

\section{definition of vector space}
\subsection{definition textbf{addition, scalar multiplication}}
\begin{itemize}
    \item An \textbf{addition} on a set V is a function that assigns an element $u + v \in V$ to each pair of elements $u ,v \in R$
    \item A \textbf{scalar multiplication} on a set V is a function that assigns an element $\lambda v \in F$ to each $v \in R$
 \end{itemize}
    \subsubsection{definition vector space}
    A \textbf{vector space} is a set V along with addition on V and a scalar multiplication on V sunch taht following Properties hold: \\
    \textbf{commutativity}
    \begin{align*}
       \notag u + v = v + u \text{ for all } u,v \in V;
    \end{align*}
    \textbf{associativity}
    \begin{align*}
       \notag (u + v) + w = v + (u + w) \text{ and } (ab)v = a(bv) \text{ for all } u,v ,w \in V \text{,and all } a,v \in F;
    \end{align*}
    \textbf{additive identity} \\
    there exists an element 0 $\in R$ sunch that $v + 0 = v$ for all $v \in V$. \\
    \textbf{additive inverse} \\
    for every $v \in V$, there exists $w \in R$ sunch that $v + w = 0 $\\
    \textbf{multiplicative indentity}\\
    $1v = v$ for all $v \in V$; \\
    \textbf{distribution Properties} \\
    $a(u + v) = au + av$ and $(a + b)u = au + bu$ for all $a, b \in F$ and all $u, v \in V$

\section{\textbf{Subspaces}}
    \subsection{definition subspaces}
    A subset U of V is called a \textbf{subspaces} of V if U is also a vector space (using the same addition and scalar multiplication as on V) \\
        \subsubsection{Example}
            \begin{align*}
                \{(x_1, x_2, 0) : x_1, x_2 \in F\} \text{is a subspace of } F^3
            \end{align*}
    \subsection{\textbf{\color{cyan} Condiitions for a subspaces}}
    A subset of U of V is subspace of V if and noly if U statisfies the following three Condiitions:\\
    \textbf{additive indentity}
    \begin{align*}
        0 \in U
    \end{align*}
    \textbf{closed under addition}
    \begin{align*}
        u,w \in U \text{ implies } u + w \in U
    \end{align*}
    \textbf{closed under scalar multiplication}
    \begin{align*}
        a \in F \text{ and } u \in U \text{ implies } au \in U
    \end{align*}
    \subsection{\textbf{\color{cyan} Sum of subspace}}
        \subsubsection{definition \textbf{sum of subsets}}
        Suppose $U_1,...,U_m$ are subsets of V. the sum of $U_1,...,U_m$, denoted $U_1,...,U_m$, is the set of all possible sum of elements
        of $U_1,...,U_m$ More precisely,
        \begin{align*}
            U_1 + ... + U_m = \{u_1 + ... + u_m : u_1\in U_1,..., u_m \in U_m\}.
        \end{align*}
        \subsubsection{\textbf{\color{cyan} sum of subsoace is the smallest containing subspace}}
        Suppose $U_1,...U_m$ are the subspaces of V. Then $U_1 + ... + U_m$ is the smallest subspace of V containing $U_1,...,U_m$ 
        \subsubsection{\textbf{\color{cyan} diretct sums}}
        Suppose $U_1,...,U_m$ are subspace of V. Every element of $U_1,...,U_m$ can be written in the form
        \begin{align*}
            u_1 + ... + u_m,
        \end{align*}
        Where each $u_j$ is in $U_j$. We will be especially interested in cases where each vector in $U_1,...,U_m$ can represented in the form above
        in only one way. this situation is so important that we give it a special name:direct sum.  
        \subsubsection{\textbf{\color{cyan} condititons for a direct sum}}
        Suppose $U_1,...,U_m$ are subspaces of V. then $U_1 + ... + U_m$ is a direct sum if and only way to wirite 0 as sum $u_1 + ... +u_m$,
        where each $u_j$ is in $U_j$, is by taking each $u_j$ equal to 0.
        \subsubsection{\textbf{\color{cyan} direct sum of two subspace}}
        Suppose U and W are subspace of V. then U + W is a direct sum if and only $U \cap V = \{0\}$

\end{document}
